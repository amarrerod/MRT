%% This is file `elsarticle-template-1-num.tex',
%%
%% Copyright 2009 Elsevier Ltd
%%
%% This file is part of the 'Elsarticle Bundle'.
%% ---------------------------------------------
%%
%% It may be distributed under the conditions of the LaTeX Project Public
%% License, either version 1.2 of this license or (at your option) any
%% later version.  The latest version of this license is in
%%    http://www.latex-project.org/lppl.txt
%% and version 1.2 or later is part of all distributions of LaTeX
%% version 1999/12/01 or later.
%%
%% Template article for Elsevier's document class `elsarticle'
%% with numbered style bibliographic references
%%
%% $Id: elsarticle-template-1-num.tex 149 2009-10-08 05:01:15Z rishi $
%% $URL: http://lenova.river-valley.com/svn/elsbst/trunk/elsarticle-template-1-num.tex $
%%
\documentclass[preprint,14pt]{elsarticle}

%% Use the option review to obtain double line spacing
%% \documentclass[preprint,review,12pt]{elsarticle}

%% Use the options 1p,twocolumn; 3p; 3p,twocolumn; 5p; or 5p,twocolumn
%% for a journal layout:
%% \documentclass[final,1p,times]{elsarticle}
%% \documentclass[final,1p,times,twocolumn]{elsarticle}
%% \documentclass[final,3p,times]{elsarticle}
%% \documentclass[final,3p,times,twocolumn]{elsarticle}
%% \documentclass[final,5p,times]{elsarticle}
%% \documentclass[final,5p,times,twocolumn]{elsarticle}

%% The graphicx package provides the includegraphics command.
\usepackage{graphicx}
\usepackage[dvips]{epsfig}
\usepackage[utf8]{inputenc}
\usepackage[spanish]{babel}
\usepackage{alltt}
\usepackage{hyperref}

\journal{University of La Laguna}

\begin{document}

\begin{frontmatter}

%% Title, authors and addresses

\title{\textbf{Tourist Trip Route Problem}}
\author{Alejandro Marrero Díaz \\ email {alu0100825008@ull.edu.es}}
\address{La Laguna, Tenerife, ES}

\begin{abstract}
%% Text of abstract
WRITE SOME USELESS SHIT HERE
\end{abstract}
\begin{keyword}
Computing Science \sep Tourist Trip Route Problem \sep Metaheuristics
\end{keyword}
\end{frontmatter}


\section{Data Collection Process}
\label{S:data}
The process of collecting data was pretty rudimentary. Simply, I just select thirty different places in Tenerife from the web \href{https://www.tripadvisor.es/Attractions-g187479-Activities-Tenerife_Canary_Islands.html}{TripAdvisor} and, after that I just calculated the distance between each place with the Google Maps Tool. Due to the fact that this work is only for educational purpose, I assume that the distance between to places A and B where the same, so, the distance from A to B is the same that the distance from B to A. And, with this, I am able to store the distances into a unidimensional array doing the compute process easier.

\section{Initial Solution Generation}
All the implemented algorithms must start from any initial feasible solution, so, for this work I have considered two ways of create that initial solution. \\
On the one hand, I have developed a Greedy algorithm which creates an initial solution considering only the stars of every place. \\
On the other hand, a random solution was created considering a weighing equation \label{eq:w} an applying the Opposite-Based Learning technique.
\label{S:random}
\subsection{Opposition-Based Learning}
\label{ss:OBL}
Opposition-based Learning (OBL) \cite{obl, obl2, OPSO, OPSO2} is a computing concept which has demostrated great efectivity at the time of improve several optimization algorithms. When we are evaluating a solution X, which belongs to the set of feasibles solutions S, simultaneously, we calculate the opposite solution $\overline{X}$, in order to achieve a better exploration of the search space $\Omega$ looking for the global optima  \cite{obl}.

Being $x \in \Re $ a real number defined within a certain range $x \in [a,b]$. The opposite number of X, denoted as $\overline{x}$ is defined as follows \cite{obl}: \\
\begin{equation}\label{eq:w}
     \overline{x} = a + b - x  \\
 \end{equation}

 Taking into acount that, in this problem we are not working with real numbers but places, the way we calculate the opposite place of any X place is the same but, in this case, the range $[a, b]$ is the number of possible places we can select in the problem.

Finally, the place inserted in the initial solution is the one which has better rate when aplying the following equation: \\

\begin{equation}
    \frac{0.3 * X_{duration} + 0.7 * X_{stars}}{X_{stars}^{2}}    \\
 \end{equation}

\section{Algorithms and Results}
\label{S:algorithms}
WRITE SOME USELESS SHIT HERE

\section{Encountered Problems}
\label{S:problems}

WRITE SOME USELESS SHIT HERE

\section{Bibliography}
\addcontentsline{toc}{chapter}{Bibliografía}
\bibliographystyle{plain}

\bibliography{VRP}

\end{document}
